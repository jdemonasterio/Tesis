

%------------------------------------+
% Definition of theorem environments |
%------------------------------------+
% Declare theorem styles that remove final dot and use bold font for notes
\newtheoremstyle{plaindotless}{\topsep}{\topsep}{\itshape}{0pt}{\bfseries}{}%
{5pt plus 1pt minus 1pt}{\thmname{#1}\thmnumber{ #2}\bfseries{\thmnote{ (#3)}}}
\newtheoremstyle{definitiondotless}{\topsep}{\topsep}{\normalfont}{0pt}%
{\bfseries}{}{5pt plus 1pt minus 1pt}%
{\thmname{#1}\thmnumber{ #2}\bfseries{\thmnote{ (#3)}}}
\newtheoremstyle{remarkdotless}{0.5\topsep}{0.5\topsep}{\normalfont}{0pt}%
{\itshape}{}{5pt plus 1pt minus 1pt}%
{\thmname{#1}\normalfont\thmnumber{ #2}\itshape{\thmnote{ (#3)}}}
%
%% Define style dependent environments and number them consecutively per section
\theoremstyle{plaindotless}
\newtheorem{theorem}{Theorem}[section]
\newtheorem*{theorem*}{Theorem.}
\newtheorem{proposition}[theorem]{Proposition}
\newtheorem*{proposition*}{Proposition.}
\newtheorem{lemma}[theorem]{Lemma}
\newtheorem*{lemma*}{Lemma.}
\newtheorem{corollary}[theorem]{Corollary}
\newtheorem*{corollary*}{Corollary.}

\theoremstyle{definitiondotless}
\newtheorem{definition}[theorem]{Definition}
\newtheorem*{definition*}{Definition.}
\newtheorem{examplex}[theorem]{Example}
\newtheorem*{examplestarred}{Example.}
\newtheorem*{continuedex}{Example \continuedexref\space Continued.}
\newtheorem{exercise}[theorem]{Exercise}
\newtheorem*{exercise*}{Exercise.}
\newtheorem*{solution*}{Solution.}
\newtheorem{problem}{Problem}

\theoremstyle{remarkdotless}
\newtheorem{remark}[theorem]{Remark}
\newtheorem*{remark*}{Remark.}
\newtheorem*{notation*}{Notation.}

% Define numbered, unnumbered and continued examples with triangle end mark
\newcommand{\myqedsymbol}{\ensuremath{\triangle}}

\newenvironment{example}
  {\pushQED{\qed} \renewcommand{\qedsymbol}{\myqedsymbol}\examplex}
  {\popQED\endexamplex}

\newenvironment{example*}
  {\pushQED{\qed}\renewcommand{\qedsymbol}{\myqedsymbol}\examplestarred}
  {\popQED\endexamplestarred}

\newenvironment{examcont}[1]
  {\pushQED{\qed}\renewcommand{\qedsymbol}{\myqedsymbol}%
    \newcommand{\continuedexref}{\ref*{#1}}\continuedex}
  {\popQED\endcontinuedex}


%-----------------------------------------------+
% Cross-references settings (cleveref settings) |
%-----------------------------------------------+

\crefname{exercise}{Exercise}{Exercises}
\crefname{enumerate}{Enumeration}{Enumerations}
\crefname{stepsi}{Step}{Steps}
\crefname{problem}{Problem}{Problems}
\crefname{appsec}{Appendix}{Appendices}

% Uncommenting this brings problems when referrencing equations.
%\crefname{equation}{}{}
%\crefformat{equation}{#2(#1)#3}
%\crefrangeformat{equation}{#3(#1)#4 to #5(#2)#6}
%\crefmultiformat{equation}{#2(#1)#3}{ and #2(#1)#3}{, #2(#1)#3}{ and #2(#1)#3}
%\crefrangemultiformat{equation}{#3(#1)#4 to #5(#2)#6}{ and #3(#1)#4 to #5(#2)#6}%
%{, #3(#1)#4 to #5(#2)#6}{ and #3(#1)#4 to #5(#2)#6}


%-----------------------------------------------+
% Citation settings (natbib settings)           |
%-----------------------------------------------+
%\usepackage[sort, numbers]{natbib}
% Set citation style
%\setcitestyle{square}

%-------------------+
% Table of contents |
%-------------------+
% Add bookmark for table of contents and increase spacing of items
% \preto{\tableofcontents}{\cleardoublepage\pdfbookmark[0]{\contentsname}{toc}%  \setstretch{1.1}}
\preto{\tableofcontents}{\clearpage\pdfbookmark[0]{\contentsname}{toc}%
  \setstretch{1.1}}
\appto{\tableofcontents}{\singlespacing}

%---------------------------------------------------+
% (Re)Definition of new commands and math operators |
%---------------------------------------------------+
% Numbers
\DeclareMathOperator{\N}{\mathbb{N}}
\DeclareMathOperator{\Z}{\mathbb{Z}}
\DeclareMathOperator{\Q}{\mathbb{Q}}
\DeclareMathOperator{\R}{\mathbb{R}}
% Probability
\DeclareMathOperator{\E}{\mathbb{E}}
\DeclareMathOperator{\Expect}{\mathbb{E}}
\DeclareMathOperator{\Var}{\mathrm{Var}}
\DeclareMathOperator{\Cov}{\mathrm{Cov}}
% Delimiters
\DeclarePairedDelimiter{\abs}{\lvert}{\rvert}
\DeclarePairedDelimiter{\norm}{\lvert\lvert}{\rvert\rvert}
% Miscellaneous
\renewcommand{\d}{\ensuremath{\operatorname{d}\!}}  % Differential
\renewcommand{\L}{\ensuremath{\operatorname{\mathcal{L}}}}  % Lagrangian
\DeclareMathOperator{\calN}{\mathcal{N}}
\DeclareMathOperator{\calG}{\mathcal{G}}
\DeclareMathOperator{\calL}{\mathcal{L}}
\DeclareMathOperator{\calE}{\mathcal{E}}
% argmin and max operators
\DeclareMathOperator*{\argmin}{argmin} % no space, limits underneath in displays
% \DeclareMathOperator{\argmin}{argmin} % no space, limits on side in displays
\DeclareMathOperator*{\argmax}{argmax} % no space, limits underneath in displays
% \DeclareMathOperator{\argmax}{argmax} % no space, limits on side in displays



\newcommand\MyBox[2]{
	\fbox{\lower0.75cm
		\vbox to 1.7cm{\vfil
			\hbox to 1.7cm{\hfil\parbox{1.4cm}{#1\\#2}\hfil}
			\vfil}%
	}%
}
